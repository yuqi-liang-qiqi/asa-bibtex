% Before compiling this example, you must generate the style files from the source:
%     latex biblatex-asa.ins
% This will produce biblatex-asa.sty, asa.bbx, and asa.cbx in the same directory.
% Only after this step can you compile this example successfully.
\documentclass{article}
\usepackage[utf8]{inputenc}
\usepackage[T1]{fontenc}
\usepackage[backend=biber]{biblatex-asa}

% Add your bibliography file
\addbibresource{example.bib}

\title{Using the biblatex-asa Package: A Demonstration}
\author{Example Author}
\date{\today}

\begin{document}

\maketitle

\section{Introduction}

This document demonstrates the usage of the \texttt{biblatex-asa} package, which provides American Sociological Association (ASA) citation and bibliography formatting for LaTeX documents.

\section{Citation Examples}

\subsection{Parenthetical Citations}

Basic parenthetical citations using \verb|\parencite|:
\begin{itemize}
    \item Single author: \parencite{brown2022}
    \item Two authors: \parencite{kao2003}
    \item Multiple citations: \parencite{brown2022,kao2003,smith2020a}
\end{itemize}

\subsection{Text Citations}

In-text citations using \verb|\textcite|:
\begin{itemize}
    \item \textcite{brown2022} argues that methodology is crucial.
    \item \textcite{kao2003} provide empirical evidence.
    \item According to \textcite{smith2020a}, recent developments suggest new directions.
\end{itemize}

\subsection{Different Entry Types}

The package handles various types of sources:
\begin{itemize}
    \item Journal articles: \parencite{brown2022}
    \item Books: \parencite{johnson2019}
    \item Book chapters: \parencite{wilson2021}
    \item Organizational reports: \parencite{WHO2022}
    \item Miscellaneous sources: \parencite{ASA1997}
\end{itemize}

\section{Package Options}

The \texttt{biblatex-asa} package supports several options:

\begin{itemize}
    \item \texttt{giveninits}: Use initials for given names
    \item \texttt{nogiveninits}: Use full given names (default)
    \item \texttt{backend=biber}: Use biber backend (recommended)
    \item \texttt{backend=bibtex}: Use bibtex backend
\end{itemize}

Example usage with options:
\begin{verbatim}
\usepackage[giveninits,backend=biber]{biblatex-asa}
\end{verbatim}

\section{Compilation}

To compile this document:
\begin{enumerate}
    \item \texttt{pdflatex example-with-package.tex}
    \item \texttt{biber example-with-package}
    \item \texttt{pdflatex example-with-package.tex}
    \item \texttt{pdflatex example-with-package.tex}
\end{enumerate}

\section{Conclusion}

The \texttt{biblatex-asa} package provides a convenient way to format citations and bibliographies according to ASA standards. It combines the power of biblatex with ASA-specific formatting requirements.

\nocite{*} % Include all bibliography entries, even if not cited

% ----------------- In-text citation tests (added by AI) -----------------

\section*{In-text Citation Examples}

% Single author
As \textcite{bernard1957} argues, experimental medicine is foundational.

% Two authors
Urbanization trends are discussed by \textcite{lee2021a} and \textcite{lee2021b}.

% Three authors
\textcite{deschenes2000} provide a computer file resource for supervised release studies.

% Group/organization author
According to \textcite{who2022}, global health is a major concern.

% n.d. (no date)
Mystery remains about the unknown year \parencite{doe_nodate}.

% Forthcoming
The gender significance of class is explored in \textcite{szelenyiForthcoming}.

% a/b/c year disambiguation
Goodman analyzed latent structure in 1947 \parencite{goodman1947a, goodman1947b}.

% Page number citation
See \parencite[pp.~63--93]{sampson1992} for a discussion of family management.

% Multiple citations
Several works address this topic \parencite{brown2022,kao2003,smith2020a}.

% Citing a dataset
See dataset \textcite{charles1990} for more details.

% Citing a government report
See \textcite{usbc1960} for census data.

% Citing a website
See \textcite{asa1997} for the ASA action alert.

% End of in-text citation tests

\printbibliography[title={REFERENCES}]

\end{document} 