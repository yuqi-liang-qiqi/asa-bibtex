\documentclass[11pt,a4paper]{ltxdoc}
\usepackage[utf8]{inputenc}
\usepackage[T1]{fontenc}
\usepackage{lmodern}
\usepackage{microtype}
\usepackage{xcolor}
\usepackage{booktabs}
\usepackage{array}
\usepackage{listings}
\usepackage{fancyvrb}
\usepackage{url}

% Define custom blue color
\definecolor{customblue}{HTML}{4169E1}

% Configure hyperref with custom colors
\usepackage[
  colorlinks=true,
  linkcolor=customblue,
  urlcolor=customblue,
  citecolor=customblue,
  filecolor=customblue,
  bookmarksnumbered=true,
  bookmarksopen=true
]{hyperref}

% Package being documented
\usepackage[backend=biber]{biblatex}
\addbibresource{../tests/test-asa.bib}

\lstset{
  basicstyle=\ttfamily\small,
  commentstyle=\color{gray},
  keywordstyle=\color{blue},
  stringstyle=\color{red},
  showstringspaces=false,
  breaklines=true,
  frame=single,
  backgroundcolor=\color{gray!10}
}

\title{The \textsf{biblatex-asa} Package\\
       \Large American Sociological Association Style for \textsf{biblatex}}
\author{Yuqi Liang\\
        \texttt{yuqi.liang.1900@gmail.com}}
\date{Version 1.0.0\\July 3, 2025}

\begin{document}

\maketitle

\begin{abstract}
The \textsf{biblatex-asa} package provides a complete implementation of the American Sociological Association (ASA) citation and bibliography style for LaTeX using the \textsf{biblatex} package. This implementation follows the format requirements of the \emph{American Sociological Review}, the flagship journal of the American Sociological Association and one of the most prestigious sociology publications worldwide, ensuring compatibility with the highest academic standards in the field.
\end{abstract}

\tableofcontents

\section{Introduction}

The \textsf{biblatex-asa} package provides citation and bibliography formatting according to the American Sociological Association (ASA) style guide. This package is particularly useful for:

\begin{itemize}
  \item Students writing sociology papers or theses
  \item Researchers submitting to sociology journals, especially ASA journals such as \emph{American Sociological Review}, \emph{American Journal of Sociology}, \emph{Social Psychology Quarterly}, and others
  \item Anyone requiring ASA-style citations and references
\end{itemize}

\subsection{Key Features}

\begin{itemize}
  \item Complete ASA citation and bibliography formatting
  \item Support for all major entry types (articles, books, datasets, etc.)
  \item Proper handling of multiple authors and publication dates
  \item Correct page number formatting
  \item Self-contained implementation with no external style dependencies (works out-of-the-box without requiring additional style packages or external bibliography style files)
  \item Full integration with \textsf{biblatex} (leverages biblatex's powerful features including Unicode support, advanced sorting, flexible formatting options, and superior handling of complex bibliographic data compared to traditional BibTeX)
\end{itemize}

\subsection{Understanding Reference Systems: A Beginner's Guide}

If you're new to LaTeX or academic writing, this section will help you understand the basics of referencing systems and why we use \textsf{biblatex}.

\subsubsection{What Are Referencing Styles?}

In academic writing, you need to cite your sources in two places:

\begin{enumerate}
  \item \textbf{In-text citations}: Brief references within your text that point to full source information
  \item \textbf{Bibliography/References}: Complete source information listed at the end of your paper
\end{enumerate}

Different academic fields use different formatting rules for these citations. Here are the major referencing styles and their typical academic disciplines:

\begin{description}
  \item[\textbf{ASA (American Sociological Association)}] \hfill \\
  \textbf{Common disciplines:} Sociology \\
  \textbf{Format features:} Author surname + year, full names in bibliography, journal names italicized, volume(issue):pages \\
  \textbf{Example:} In-text: (Smith 2020) → Bibliography: Smith, John. 2020. "Title." \emph{Journal} 15(3):100-110.
  
  \textbf{Important note:} While ASA is the standard for American Sociological Association journals, not all sociology journals use ASA format. For example:
  \begin{itemize}
    \item \emph{Sociological Science} explicitly requires ASA format, following the ASA Style Guide strictly for both in-text citations and bibliography.
    \item \emph{Social Forces} uses Chicago Manual of Style author-date system instead of ASA format, with citations like (Lipset 1960) and different bibliography formatting rules.
    \item Always check individual journal guidelines, as sociology journals may have varying requirements.
  \end{itemize}
  
  \item[\textbf{APA (American Psychological Association)}] \hfill \\
  \textbf{Common disciplines:} Psychology, Education, Social Sciences \\
  \textbf{Format features:} Author surname, initial + year, sentence case titles, journal names italicized, volume(issue), pages \\
  \textbf{Example:} In-text: (Smith, 2020) → Bibliography: Smith, J. (2020). Title. \emph{Journal}, \emph{15}(3), 100-110.
  
  \item[\textbf{Chicago Manual of Style}] \hfill \\
  \textbf{Common disciplines:} History, Literature, Philosophy, Humanities \\
  \textbf{Format features:} Two systems (author-date and notes-bibliography); shown here is author-date \\
  \textbf{Example:} In-text: (Smith 2020, 100) → Bibliography: Smith, John. "Title." \emph{Journal} 15, no. 3 (2020): 100-110.
  
  \item[\textbf{Harvard Referencing}] \hfill \\
  \textbf{Common disciplines:} Business, Management, Economics, Social Sciences (esp. in Commonwealth countries) \\
  \textbf{Format features:} Author surname, initial + year, titles in single quotes, journal names italicized, volume(issue), pp. pages \\
  \textbf{Example:} In-text: (Smith 2020) → Bibliography: Smith, J. (2020) 'Title', \emph{Journal}, \emph{15}(3), pp. 100-110.
  
  \item[\textbf{Economics Style}] \hfill \\
  \textbf{Common disciplines:} Economics (varies by journal) \\
  \textbf{Format features:} Author-year system, full author names, titles in quotation marks, journal names italicized \\
  \textbf{Example:} In-text: Smith (2020) → Bibliography: Smith, John (2020). "Title". \emph{Journal}, 15(3), 100-110.
\end{description}

\subsubsection{BibTeX vs. biblatex: Why the Difference Matters}

LaTeX has two main systems for handling references:

\begin{description}
  \item[\textbf{BibTeX (the older system):}] \hfill \\
  \begin{itemize}
    \item Uses fixed style files (like \texttt{.bst}) that are hard to customize
    \item Limited Unicode support (problems with accents, non-Latin scripts)
    \item Simpler but less flexible citation commands
    \item Example: \texttt{\textbackslash cite\{smith2020\}} → [1] or (Smith, 2020)
  \end{itemize}
  
  \item[\textbf{biblatex (the modern system):}] \hfill \\
  \begin{itemize}
    \item Highly customizable styles written in LaTeX itself
    \item Full Unicode support (handles any language, accents, symbols)
    \item More sophisticated citation commands for different contexts
    \item Better handling of complex bibliographic data (DOIs, URLs, datasets, etc.)
    \item Example: Multiple commands like \texttt{\textbackslash parencite\{\}}, \texttt{\textbackslash textcite\{\}}, etc.
  \end{itemize}
\end{description}

\subsubsection{Why We Use biblatex for ASA Style}

\textsf{biblatex} is the better choice for ASA formatting because:

\begin{itemize}
  \item \textbf{Precision}: ASA has specific formatting rules (like "Pp. 63–93" for book chapters) that are easier to implement in biblatex
  \item \textbf{Flexibility}: Different citation contexts need different formats—biblatex provides specialized commands for each
  \item \textbf{Modern features}: ASA papers often cite online sources, datasets, and other modern materials that biblatex handles better
  \item \textbf{Reliability}: Once set up correctly, biblatex automatically formats everything consistently
\end{itemize}

\subsubsection{Why Different Citation Commands?}

In academic writing, you cite sources in different ways depending on the context:

\begin{itemize}
  \item \textbf{Supporting evidence}: "Studies show this trend \texttt{\textbackslash parencite\{smith2020\}}" → "Studies show this trend (Smith 2020)."
  \item \textbf{Introducing an author}: "\texttt{\textbackslash textcite\{jones2019\}} argues that..." → "Jones (2019) argues that..."
  \item \textbf{Multiple sources}: "\texttt{\textbackslash parencite\{a2020,b2021,c2022\}}" → "(A 2020; B 2021; C 2022)"
\end{itemize}

Each command produces the correct formatting for its context automatically. This is much better than manually typing "(Smith 2020)" because:
\begin{itemize}
  \item If you change the bibliography database, citations update automatically
  \item If you change style guides (ASA → APA), all citations reformat automatically  
  \item The system prevents inconsistencies and typos
\end{itemize}

\subsubsection{How This Package Helps}

The \textsf{biblatex-asa} package handles all the complexity for you:
\begin{itemize}
  \item You write: \texttt{\textbackslash parencite\{smith2020\}}
  \item You get: (Smith 2020) in perfect ASA format
  \item The bibliography automatically formats as: Smith, John. 2020. "Article Title." Journal Name 15(3):100-110.
\end{itemize}

\textbf{Bottom line}: You focus on writing your research, and the package ensures your citations follow ASA style perfectly.

\section{Installation}

\subsection{Package Requirements}

You will need to be using a recent LaTeX distribution and the following packages:

\begin{itemize}
  \item \textbf{LaTeX distribution:} TeX Live 2020 or later, MiKTeX 2020 or later
  \item \textbf{biblatex:} Version 3.14 or later (provides the underlying bibliography system)
  \item \textbf{biber:} Version 2.14 or later (recommended backend for processing bibliography data)
  \item \textbf{etoolbox:} Version 2.5 or later (automatically loaded by biblatex, provides programming facilities)
\end{itemize}

\textbf{Important notes:}
\begin{itemize}
  \item The \textsf{biblatex-asa} package is built on top of the \texttt{standard} and \texttt{authoryear} styles from biblatex
  \item While bibtex backend is supported, \textbf{biber is strongly recommended} for full functionality and better Unicode support
  \item If you're using an older LaTeX distribution, some features may not work correctly
  \item The package automatically loads biblatex with appropriate ASA-specific options
\end{itemize}

\subsection{Installation Methods}

\subsubsection{Automatic Installation (Recommended)}

For most users, the package can be installed automatically through your TeX distribution's package manager:

\begin{itemize}
  \item \textbf{Overleaf (Easiest for Beginners):} The package is pre-installed and ready to use. Simply add \texttt{\textbackslash usepackage\{biblatex-asa\}} to your document preamble—no download or installation needed! Overleaf automatically handles all package management for you.

  \item \textbf{TeX Live}: The package is available through \texttt{tlmgr}
  \begin{verbatim}
  tlmgr install biblatex-asa
  \end{verbatim}
  
  \item \textbf{MiKTeX}: The package is available through the MiKTeX Package Manager or automatically installed when first used
\end{itemize}

\subsubsection{Manual Installation}

If automatic installation is not available or you need the latest development version:

\begin{enumerate}
  \item Download the package files from CTAN or GitHub
  \item For local use in a single project, copy these files to your project directory:
  \begin{itemize}
    \item \texttt{biblatex-asa.sty}
    \item \texttt{asa.bbx}
    \item \texttt{asa.cbx}
  \end{itemize}
  \item For system-wide installation, place the files in your local \texttt{texmf} tree:
  \begin{verbatim}
  texmf/tex/latex/biblatex-asa/
  \end{verbatim}
  \item Update your filename database (required so LaTeX can find the newly installed files):
  \begin{itemize}
    \item TeX Live: run \texttt{texhash} or \texttt{mktexlsr}
    \item MiKTeX: run \texttt{initexmf --update-fndb}
  \end{itemize}
\end{enumerate}

\section{Usage}

\subsection{Basic Usage}

To use the \textsf{biblatex-asa} package in your document:

\begin{lstlisting}[language=TeX]
\documentclass{article}
\usepackage{biblatex-asa}
\addbibresource{references.bib}

\begin{document}
This is a citation \parencite{key2023}.
\textcite{author2022} argues that...
\printbibliography
\end{document}
\end{lstlisting}

\subsection{Direct Style Usage}

Alternatively, you can use the ASA style directly with \textsf{biblatex}:

\begin{lstlisting}[language=TeX]
\usepackage[backend=biber,style=asa]{biblatex}
\addbibresource{references.bib}
\end{lstlisting}

\subsection{Package Options}

The \textsf{biblatex-asa} package accepts the following options:

\begin{description}
  \item[\texttt{giveninits}] Use initials for given names
  \item[\texttt{nogiveninits}] Use full given names (default)
  \item[\texttt{backend=biber}] Use biber backend (recommended)
  \item[\texttt{backend=bibtex}] Use bibtex backend
\end{description}

Example with options (note: square brackets \texttt{[]} contain optional parameters, curly braces \texttt{\{\}} contain the package name):
\begin{lstlisting}[language=TeX]
\usepackage[giveninits,backend=biber]{biblatex-asa}
\end{lstlisting}

\textbf{LaTeX Syntax Explanation:}
\begin{itemize}
  \item \textbf{Square brackets \texttt{[options]}:} Optional settings that modify how the package behaves
  \item \textbf{Curly braces \texttt{\{package-name\}}:} Required package name that must be loaded
  \item Multiple options are separated by commas within the square brackets
\end{itemize}

\section{Citation Commands}

\subsection{Standard Commands}

The package supports all standard \textsf{biblatex} citation commands. Here are detailed examples of when and how to use each:

\begin{description}
  \item[\cs{parencite\{key\}}] Parenthetical citation: (Author Year)
  
  \textbf{When to use:} At the end of a sentence or when supporting a claim.
  
  \textbf{Examples:}
  \begin{itemize}
    \item Studies show increasing social inequality \texttt{\cs{parencite\{smith2020\}}} → Studies show increasing social inequality (Smith 2020).
    \item Research supports this theory \texttt{\cs{parencite\{jones2019,brown2021\}}} → Research supports this theory (Jones 2019; Brown 2021).
    \item For page-specific references: \texttt{\cs{parencite[p.~42]\{doe2018\}}} → (Doe 2018, p. 42).
  \end{itemize}
  
  \item[\cs{textcite\{key\}}] In-text citation: Author (Year)
  
  \textbf{When to use:} When the author's name is part of your sentence.
  
  \textbf{Examples:}
  \begin{itemize}
    \item \texttt{\cs{textcite\{wilson2022\}} argues that social media impacts behavior} → Wilson (2022) argues that social media impacts behavior.
    \item \texttt{According to \cs{textcite\{garcia2020\}}, urban development follows predictable patterns} → According to Garcia (2020), urban development follows predictable patterns.
    \item \texttt{\cs{textcite[pp.~15--20]\{taylor2019\}} provides detailed analysis} → Taylor (2019, pp. 15--20) provides detailed analysis.
  \end{itemize}
  
  \item[\cs{cite\{key\}}] Bare citation: Author Year (no parentheses)
  
  \textbf{When to use:} Rarely used in ASA style. Only for special formatting needs where you want to control punctuation manually.
  
  \textbf{Examples:}
  \begin{itemize}
    \item In narrative text: \texttt{The \cs{cite\{smith2020\}} findings show...} → The Smith 2020 findings show...
    \item \textbf{Recommendation:} Use \texttt{\cs{parencite}} or \texttt{\cs{textcite}} instead for standard ASA citations.
  \end{itemize}
  

\end{description}



\subsection{Advanced Citation Examples}

\textbf{Common Usage Patterns:}

\begin{enumerate}
  \item \textbf{Introducing an author's argument:}
  \begin{itemize}
    \item \texttt{\cs{textcite\{author2023\}} argues that...} → Author (2023) argues that...
    \item \texttt{According to \cs{textcite\{researcher2022\}}, the data shows...} → According to Researcher (2022), the data shows...
  \end{itemize}
  
  \item \textbf{Supporting your claims:}
  \begin{itemize}
    \item \texttt{This trend is well-documented \cs{parencite\{study2021\}}.} → This trend is well-documented (Study 2021).
    \item \texttt{Multiple studies confirm this pattern \cs{parencite\{one2020,two2021,three2022\}}.} → Multiple studies confirm this pattern (One 2020; Two 2021; Three 2022).
  \end{itemize}
  
  \item \textbf{Citing specific pages:}
  \begin{itemize}
    \item \texttt{\cs{parencite[p.~25]\{book2019\}}} → (Book 2019, p. 25)
    \item \texttt{\cs{textcite[pp.~100--105]\{article2020\}}} → Article (2020, pp. 100--105)
  \end{itemize}
  
  \item \textbf{Distinguishing same-year publications:}
  \begin{itemize}
    \item \texttt{\cs{parencite\{smith2020a,smith2020b\}}} → (Smith 2020a, 2020b)
    \item The package automatically handles year disambiguation with a, b, c suffixes.
  \end{itemize}
\end{enumerate}

\textbf{Quick Reference - When to Use Which Command:}

\begin{center}
\begin{tabular}{ll}
\hline
\textbf{Situation} & \textbf{Command} \\
\hline
End of sentence, supporting claim & \texttt{\cs{parencite}} \\
Author name is part of your sentence & \texttt{\cs{textcite}} \\
Special formatting (rarely used) & \texttt{\cs{cite}} \\
\hline
\end{tabular}
\end{center}

\textbf{Important Note:} ASA style uses the author-date citation system. All citations should be in-text using the format: (Author Year) or Author (Year). Footnote citations are not used in ASA style.

\section{Compilation}

\subsection{Overleaf Compilation (Recommended for Beginners)}

If you're using Overleaf, the compilation process is much simpler:

\begin{itemize}
  \item Overleaf automatically handles the compilation sequence for you
  \item Simply click the \textbf{Recompile} button and Overleaf will run the necessary commands in the correct order
  \item Make sure your project settings use:
  \begin{itemize}
    \item \textbf{Compiler:} pdfLaTeX
    \item \textbf{TeX Live version:} 2020 or later (recommended)
  \end{itemize}
  \item You can access these settings via \texttt{Menu > Settings > Compiler}
\end{itemize}

\subsection{Local Compilation (Command Line)}

To compile a document using \textsf{biblatex-asa} on your local machine, you need to run the following commands in your terminal/command prompt. Make sure you are in the same directory as your \texttt{.tex} file:

\begin{enumerate}
  \item \texttt{pdflatex document.tex}
  
  \textbf{What this does:} Compiles your LaTeX document into a PDF for the first time. This creates auxiliary files needed for the bibliography, but the references won't show up yet.
  
  \item \texttt{biber document}
  
  \textbf{What this does:} Processes your bibliography file (\texttt{.bib}) and creates the formatted reference list according to ASA style. Note: use the filename \emph{without} the \texttt{.tex} extension.
  
  \item \texttt{pdflatex document.tex}
  
  \textbf{What this does:} Compiles the document again, now incorporating the processed bibliography. Your citations and reference list will appear.
  
  \item \texttt{pdflatex document.tex}
  
  \textbf{What this does:} Final compilation to ensure all cross-references, page numbers, and table of contents are correct.
\end{enumerate}

\textbf{Important notes:}
\begin{itemize}
  \item Replace \texttt{document} with your actual filename
  \item You must have a working LaTeX installation (TeX Live, MiKTeX, etc.)
  \item Open your terminal (Mac/Linux) or Command Prompt (Windows)
  \item Navigate to your document's folder using \texttt{cd /path/to/your/folder}
\end{itemize}

\section{ASA Format Implementation}

\subsection{Citation Format}

\begin{itemize}
  \item Single author: (Smith 2020)
  \item Two authors: (Smith and Jones 2020)
  \item Three or more authors: (Smith et al. 2020)
  \item Multiple citations: (Smith 2020; Jones 2021)
\end{itemize}

\subsection{Bibliography Format}

\begin{itemize}
  \item Authors: First author inverted (Last, First), others normal order
  \item All authors listed (no "et al." truncation)
  \item Proper handling of organizational authors
  \item Correct page number formatting
\end{itemize}

\section{Supported Entry Types}

The \textsf{biblatex-asa} package supports all standard \textsf{biblatex} entry types with proper ASA formatting:

\begin{itemize}
  \item \texttt{@article} - Journal articles
  \item \texttt{@book} - Books
  \item \texttt{@incollection} - Book chapters
  \item \texttt{@inproceedings} - Conference proceedings
  \item \texttt{@misc} - Miscellaneous sources
  \item \texttt{@online} - Online sources
  \item \texttt{@report} - Reports
  \item \texttt{@thesis} - Theses and dissertations
\end{itemize}

\section{Troubleshooting}

\subsection{Common Issues}

\begin{description}
  \item[References not showing] Ensure you run \texttt{biber} (not \texttt{bibtex}).
  
  \textbf{How to do this:}
  \begin{itemize}
    \item If you use the command line, compile your document with this sequence:
    \begin{enumerate}
      \item \texttt{pdflatex yourfile.tex}
      \item \texttt{biber yourfile}
      \item \texttt{pdflatex yourfile.tex}
      \item \texttt{pdflatex yourfile.tex}
    \end{enumerate}
    \item If you use an editor (TeXShop, TeXworks, Overleaf, VS Code with LaTeX Workshop), make sure the bibliography tool is set to \texttt{biber}, not \texttt{bibtex}. In Overleaf, go to \texttt{Menu > Settings > Compiler} and select \texttt{biber}.
  \end{itemize}
  
  \item[File not found errors] Solutions depend on your installation method.
  
  \textbf{For automatic installation (recommended):}
  \begin{itemize}
    \item \textbf{TeX Live:} Check if installed: \texttt{tlmgr list --installed \textbar\ grep biblatex-asa}
    \item If not found, install: \texttt{tlmgr install biblatex-asa}
    \item Try updating: \texttt{tlmgr update --all}
    \item \textbf{MiKTeX:} Open MiKTeX Console > Packages, refresh database, search for "biblatex-asa"
    \item \textbf{Overleaf:} Package should be pre-installed; if issues persist, contact Overleaf support
  \end{itemize}
  
  \textbf{For manual installation:}
  \begin{itemize}
    \item Local: Ensure \texttt{biblatex-asa.sty}, \texttt{asa.bbx}, \texttt{asa.cbx} are in your project directory
    \item System-wide: Files should be in \texttt{texmf/tex/latex/biblatex-asa/}
    \item Run \texttt{texhash} (TeX Live) or \texttt{initexmf --update-fndb} (MiKTeX) after installation
         \item If using \texttt{\\def\\input@path\{\{../../src/\}\}}, ensure the path is correct
  \end{itemize}
  
  \item[Encoding problems] Save all files as UTF-8.
  
  \textbf{How to do this:}
  \begin{itemize}
    \item In most editors (VS Code, Sublime Text, TeXShop, etc.), choose \texttt{Save with Encoding} or \texttt{Save As} and select UTF-8.
    \item In Overleaf, files are saved as UTF-8 by default.
    \item If unsure, open the file in your editor and check the encoding setting at the bottom of the window or in the menu.
  \end{itemize}
\end{description}

\section{License and Contact}

This work is distributed under the LaTeX Project Public License (LPPL), version 1.3c or later.

\begin{description}
  \item[Author:] Yuqi Liang, University of Oxford
  \item[Email:] \texttt{yuqi.liang.1900@gmail.com}
  \item[GitHub:] \url{https://github.com/yuqi-liang-qiqi/biblatex-asa}
\end{description}

\section{Acknowledgments}

This package was developed independently to provide comprehensive ASA formatting for the \textsf{biblatex} system. Thanks to the \textsf{biblatex} and \textsf{biber} development teams for their excellent software.

\end{document} 