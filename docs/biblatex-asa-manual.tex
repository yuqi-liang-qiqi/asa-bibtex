\documentclass[11pt,a4paper]{ltxdoc}
\usepackage[utf8]{inputenc}
\usepackage[T1]{fontenc}
\usepackage{lmodern}
\usepackage{microtype}
\usepackage{xcolor}
\usepackage{booktabs}
\usepackage{array}
\usepackage{listings}
\usepackage{fancyvrb}
\usepackage{url}

% Define custom blue color
\definecolor{customblue}{HTML}{4169E1}

% Configure hyperref with custom colors
\usepackage[
  colorlinks=true,
  linkcolor=customblue,
  urlcolor=customblue,
  citecolor=customblue,
  filecolor=customblue,
  bookmarksnumbered=true,
  bookmarksopen=true
]{hyperref}

% Package being documented
\usepackage[backend=biber]{biblatex}
\addbibresource{../tests/test-asa.bib}

\lstset{
  basicstyle=\ttfamily\small,
  commentstyle=\color{gray},
  keywordstyle=\color{blue},
  stringstyle=\color{red},
  showstringspaces=false,
  breaklines=true,
  frame=single,
  backgroundcolor=\color{gray!10}
}

\title{The \textsf{biblatex-asa} Package\\
       \Large American Sociological Association Style for \textsf{biblatex}}
\author{Yuqi Liang\\
        \texttt{yuqi.liang.1900@gmail.com}}
\date{Version 1.0.0\\July 3, 2025}

\begin{document}

\maketitle

\begin{abstract}
The \textsf{biblatex-asa} package provides a complete implementation of the American Sociological Association (ASA) citation and bibliography style for LaTeX using the \textsf{biblatex} package. This implementation follows the format requirements of the \emph{American Sociological Review} and provides native ASA formatting without requiring external style dependencies.
\end{abstract}

\tableofcontents

\section{Introduction}

The \textsf{biblatex-asa} package provides citation and bibliography formatting according to the American Sociological Association (ASA) style guide. This package is particularly useful for:

\begin{itemize}
  \item Students writing sociology papers or theses
  \item Researchers submitting to sociology journals
  \item Anyone requiring ASA-style citations and references
\end{itemize}

\subsection{Key Features}

\begin{itemize}
  \item Complete ASA citation and bibliography formatting
  \item Support for all major entry types (articles, books, datasets, etc.)
  \item Proper handling of multiple authors and publication dates
  \item Correct page number formatting
  \item No external style dependencies required
  \item Full integration with \textsf{biblatex}
\end{itemize}

\section{Installation}

\subsection{Package Requirements}

You will need to be using a recent LaTeX distribution and the following packages:

\begin{itemize}
  \item \textbf{LaTeX distribution:} TeX Live 2020 or later, MiKTeX 2020 or later
  \item \textbf{biblatex:} Version 3.14 or later (provides the underlying bibliography system)
  \item \textbf{biber:} Version 2.14 or later (recommended backend for processing bibliography data)
  \item \textbf{etoolbox:} (automatically loaded by biblatex, provides programming facilities)
\end{itemize}

\textbf{Important notes:}
\begin{itemize}
  \item The \textsf{biblatex-asa} package is built on top of the \texttt{standard} and \texttt{authoryear} styles from biblatex
  \item While bibtex backend is supported, \textbf{biber is strongly recommended} for full functionality and better Unicode support
  \item If you're using an older LaTeX distribution, some features may not work correctly
  \item The package automatically loads biblatex with appropriate ASA-specific options
\end{itemize}

\subsection{Installation Methods}

\subsubsection{Automatic Installation (Recommended)}

For most users, the package can be installed automatically through your TeX distribution's package manager:

\begin{itemize}
  \item \textbf{TeX Live}: The package is available through \texttt{tlmgr}
  \begin{verbatim}
  tlmgr install biblatex-asa
  \end{verbatim}
  
  \item \textbf{MiKTeX}: The package is available through the MiKTeX Package Manager or automatically installed when first used
  
  \item \textbf{Overleaf}: The package is pre-installed and ready to use
\end{itemize}

\subsubsection{Manual Installation}

If automatic installation is not available or you need the latest development version:

\begin{enumerate}
  \item Download the package files from CTAN or GitHub
  \item For local use in a single project, copy these files to your project directory:
  \begin{itemize}
    \item \texttt{biblatex-asa.sty}
    \item \texttt{asa.bbx}
    \item \texttt{asa.cbx}
  \end{itemize}
  \item For system-wide installation, place the files in your local \texttt{texmf} tree:
  \begin{verbatim}
  texmf/tex/latex/biblatex-asa/
  \end{verbatim}
  \item Update your filename database:
  \begin{itemize}
    \item TeX Live: run \texttt{texhash} or \texttt{mktexlsr}
    \item MiKTeX: run \texttt{initexmf --update-fndb}
  \end{itemize}
\end{enumerate}

\section{Usage}

\subsection{Basic Usage}

To use the \textsf{biblatex-asa} package in your document:

\begin{lstlisting}[language=TeX]
\documentclass{article}
\usepackage{biblatex-asa}
\addbibresource{references.bib}

\begin{document}
This is a citation \parencite{key2023}.
\textcite{author2022} argues that...
\printbibliography
\end{document}
\end{lstlisting}

\subsection{Direct Style Usage}

Alternatively, you can use the ASA style directly with \textsf{biblatex}:

\begin{lstlisting}[language=TeX]
\usepackage[backend=biber,style=asa]{biblatex}
\addbibresource{references.bib}
\end{lstlisting}

\subsection{Package Options}

The \textsf{biblatex-asa} package accepts the following options:

\begin{description}
  \item[\texttt{giveninits}] Use initials for given names
  \item[\texttt{nogiveninits}] Use full given names (default)
  \item[\texttt{backend=biber}] Use biber backend (recommended)
  \item[\texttt{backend=bibtex}] Use bibtex backend
\end{description}

Example with options:
\begin{lstlisting}[language=TeX]
\usepackage[giveninits,backend=biber]{biblatex-asa}
\end{lstlisting}

\section{Citation Commands}

\subsection{Standard Commands}

The package supports all standard \textsf{biblatex} citation commands:

\begin{description}
  \item[\cs{parencite}] Parenthetical citation: (Author Year)
  \item[\cs{textcite}] In-text citation: Author (Year)
  \item[\cs{cite}] Bare citation: Author Year
  \item[\cs{footcite}] Footnote citation
\end{description}

\subsection{Convenience Commands}

The package also provides convenience commands:

\begin{description}
  \item[\cs{asacite}] Equivalent to \cs{parencite}
  \item[\cs{asatextcite}] Equivalent to \cs{textcite}
\end{description}

\section{Compilation}

\subsection{Overleaf Compilation (Recommended for Beginners)}

If you're using Overleaf, the compilation process is much simpler:

\begin{itemize}
  \item Overleaf automatically handles the compilation sequence for you
  \item Simply click the \textbf{Recompile} button and Overleaf will run the necessary commands in the correct order
  \item Make sure your project settings use:
  \begin{itemize}
    \item \textbf{Compiler:} pdfLaTeX
    \item \textbf{TeX Live version:} 2020 or later (recommended)
  \end{itemize}
  \item You can access these settings via \texttt{Menu > Settings > Compiler}
\end{itemize}

\subsection{Local Compilation (Command Line)}

To compile a document using \textsf{biblatex-asa} on your local machine, you need to run the following commands in your terminal/command prompt. Make sure you are in the same directory as your \texttt{.tex} file:

\begin{enumerate}
  \item \texttt{pdflatex document.tex}
  
  \textbf{What this does:} Compiles your LaTeX document into a PDF for the first time. This creates auxiliary files needed for the bibliography, but the references won't show up yet.
  
  \item \texttt{biber document}
  
  \textbf{What this does:} Processes your bibliography file (\texttt{.bib}) and creates the formatted reference list according to ASA style. Note: use the filename \emph{without} the \texttt{.tex} extension.
  
  \item \texttt{pdflatex document.tex}
  
  \textbf{What this does:} Compiles the document again, now incorporating the processed bibliography. Your citations and reference list will appear.
  
  \item \texttt{pdflatex document.tex}
  
  \textbf{What this does:} Final compilation to ensure all cross-references, page numbers, and table of contents are correct.
\end{enumerate}

\textbf{Important notes:}
\begin{itemize}
  \item Replace \texttt{document} with your actual filename
  \item You must have a working LaTeX installation (TeX Live, MiKTeX, etc.)
  \item Open your terminal (Mac/Linux) or Command Prompt (Windows)
  \item Navigate to your document's folder using \texttt{cd /path/to/your/folder}
\end{itemize}

\section{ASA Format Implementation}

\subsection{Citation Format}

\begin{itemize}
  \item Single author: (Smith 2020)
  \item Two authors: (Smith and Jones 2020)
  \item Three or more authors: (Smith et al. 2020)
  \item Multiple citations: (Smith 2020; Jones 2021)
\end{itemize}

\subsection{Bibliography Format}

\begin{itemize}
  \item Authors: First author inverted (Last, First), others normal order
  \item All authors listed (no "et al." truncation)
  \item Proper handling of organizational authors
  \item Correct page number formatting
\end{itemize}

\section{Supported Entry Types}

The \textsf{biblatex-asa} package supports all standard \textsf{biblatex} entry types with proper ASA formatting:

\begin{itemize}
  \item \texttt{@article} - Journal articles
  \item \texttt{@book} - Books
  \item \texttt{@incollection} - Book chapters
  \item \texttt{@inproceedings} - Conference proceedings
  \item \texttt{@misc} - Miscellaneous sources
  \item \texttt{@online} - Online sources
  \item \texttt{@report} - Reports
  \item \texttt{@thesis} - Theses and dissertations
\end{itemize}

\section{Troubleshooting}

\subsection{Common Issues}

\begin{description}
  \item[References not showing] Ensure you run \texttt{biber} (not \texttt{bibtex}).
  
  \textbf{How to do this:}
  \begin{itemize}
    \item If you use the command line, compile your document with this sequence:
    \begin{enumerate}
      \item \texttt{pdflatex yourfile.tex}
      \item \texttt{biber yourfile}
      \item \texttt{pdflatex yourfile.tex}
      \item \texttt{pdflatex yourfile.tex}
    \end{enumerate}
    \item If you use an editor (TeXShop, TeXworks, Overleaf, VS Code with LaTeX Workshop), make sure the bibliography tool is set to \texttt{biber}, not \texttt{bibtex}. In Overleaf, go to \texttt{Menu > Settings > Compiler} and select \texttt{biber}.
  \end{itemize}
  
  \item[File not found errors] Solutions depend on your installation method.
  
  \textbf{For automatic installation (recommended):}
  \begin{itemize}
    \item \textbf{TeX Live:} Check if installed: \texttt{tlmgr list --installed \textbar\ grep biblatex-asa}
    \item If not found, install: \texttt{tlmgr install biblatex-asa}
    \item Try updating: \texttt{tlmgr update --all}
    \item \textbf{MiKTeX:} Open MiKTeX Console > Packages, refresh database, search for "biblatex-asa"
    \item \textbf{Overleaf:} Package should be pre-installed; if issues persist, contact Overleaf support
  \end{itemize}
  
  \textbf{For manual installation:}
  \begin{itemize}
    \item Local: Ensure \texttt{biblatex-asa.sty}, \texttt{asa.bbx}, \texttt{asa.cbx} are in your project directory
    \item System-wide: Files should be in \texttt{texmf/tex/latex/biblatex-asa/}
    \item Run \texttt{texhash} (TeX Live) or \texttt{initexmf --update-fndb} (MiKTeX) after installation
         \item If using \texttt{\\def\\input@path\{\{../../src/\}\}}, ensure the path is correct
  \end{itemize}
  
  \item[Encoding problems] Save all files as UTF-8.
  
  \textbf{How to do this:}
  \begin{itemize}
    \item In most editors (VS Code, Sublime Text, TeXShop, etc.), choose \texttt{Save with Encoding} or \texttt{Save As} and select UTF-8.
    \item In Overleaf, files are saved as UTF-8 by default.
    \item If unsure, open the file in your editor and check the encoding setting at the bottom of the window or in the menu.
  \end{itemize}
\end{description}

\section{License and Contact}

This work is distributed under the LaTeX Project Public License (LPPL), version 1.3c or later.

\begin{description}
  \item[Author:] Yuqi Liang, University of Oxford
  \item[Email:] \texttt{yuqi.liang.1900@gmail.com}
  \item[GitHub:] \url{https://github.com/yuqi-liang-qiqi/biblatex-asa}
\end{description}

\section{Acknowledgments}

This package was developed independently to provide comprehensive ASA formatting for the \textsf{biblatex} system. Thanks to the \textsf{biblatex} and \textsf{biber} development teams for their excellent software.

\end{document} 